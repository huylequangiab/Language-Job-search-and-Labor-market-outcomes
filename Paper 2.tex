
\documentclass[12pt,a4paper]{article}
\renewcommand{\baselinestretch}{1.5}
\usepackage{lipsum}
\usepackage{authblk}
\usepackage[top=2cm, bottom=2cm, left=2cm, right=2cm]{geometry}
\usepackage{fancyhdr}
\usepackage{graphicx, subfigure}
\usepackage{amsmath}
\usepackage[colorinlistoftodos]{todonotes}
\usepackage[colorlinks=true, allcolors=blue]{hyperref}
\usepackage{booktabs, multicol, multirow}
\usepackage{epstopdf}
\usepackage{subfigure}
\usepackage{xspace}
\usepackage{ltablex}
\usepackage[flushleft]{threeparttable}
\usepackage{adjustbox}
\usepackage{pdflscape}
\usepackage{tabu}


\pagestyle{fancy}
\begin{document}

%title and author details
\title{Coming to Germany: Language, Job search and Labor market outcomes}

\author[1]{Huy Le-Quang}
%\date{24--Oct--2017} %remove date
\footnotetext[1] {Institute for Employment Research(IAB). Email: huy.le-quang@iab.de.

I would like to thank Silke Anger and other colleagues at the Institute for Employment Research in Nuremberg for helpful comments and suggestions. Special thanks goes to Dr. Soren Wichmann for detailed instructions of calculating linguistic distance used in this study. I also gratefully acknowledge financial support from the Graduate School (GradAB) at the Institute for Employment Research . All remaining errors are my own. The views, opinions, findings and conclusions expressed in this article are strictly those of the author.} 

\maketitle

\abstract{This paper investigates the impact of German language skills on labor market outcomes of immigrants in Germany by exploring the mediation effect of job search methods. I implement the potential outcomes framework to overcome the limitations of the traditional linear structural equation models in mediation analysis. Using a large representative migration sample in Germany, I show that better German language skills improve labor market performance. In particular, immigrants who speak the host-country language well are less likely to rely on their social networks to look for jobs, which in turn increases their earnings and level of job complexity. The mediation effect of job search accounts for approximately 12 percent of the total effect, which points to the importance of using mediation analysis to explore why and how language skills exert such an influence on labor market performance.
} \\
\textbf{Keywords}: language skills, job search, labor market outcomes, immigrants \\
\textbf{JEL}: I26, J24, J30

\newpage

\section{Introduction}

Since the early 1980s, a number of studies has emerged to explore the role of host-country language skills on labor market integration of immigrants. First, language skills, as being a productive trait in itself, have significant direct effect on improving immigrants' labor market outcomes (Yao and van Ours, 2015; Isphording and Otten, 2014; Chiswick and Miller, 2011; Dustmann and Fabbri, 2003). Second, language skills, especially at the beginning of migration, also facilitate early job search and the establishment of social networks (Chiswick and Miller, 2005). Further, many studies in a range of countries find that using personal network of friends and family is a popular and effective search channel to obtain job offers (Wahba and Zenou, 2005; Bentolia et al., 2010; Pellizzari, 2010). But the effectivessness of this search method in terms of later earnings and job quality is questionable as it depends on the quality of an individual's social network composition (Battu et al., 2011; Cappellari and Tatsiramos, 2015).

Empirical evidence in the existing literature on language skills of immigrants often reports the 'total effect' of language fluency on a range of labor market outcomes. But it is not empirically clear \textbf{why} such an effect arises. This study aims to investigate the role of the pre-migration language skills on the use of social networks to find jobs, which in turn affects current earnings and the quality of jobs. In doing so, I attempt to open the black-box to disentangle how and why language proficiency exerts its influence on labor market success of immigrants. I focus on pre-migration language skills because immigrants who master the national language at the very beginning of migration could immediately find better jobs and obtain an initial advantage which puts them well ahead of other immigrants who start learning the language after migration. Further, by focusing on pre-migration language skills, I am able to decompose the total effect of language skills into indirect effect through preferred early job search channels and direct effect on subsequent labor market success.

Traditionally, the mediation analysis is often implemented under the framework of linear structural equation modelling (LSEM) (MacKinnon, 2008). LSEM framework, however, cannot generalize to non-linear relationship in the mediation and outcomes equations (Imai et al., 2010). Hence, this paper implements the potential outcomes framework to accommodate for the binary mediator (using social networks to search for job) and the ordered outcome variable (the level of job complexity as a proxy for the quality of jobs).

Further problems often emerge in measuring the effect of language skills on labor market outcomes. First, self-assessed language skills are highly prone to measurement errors. Dustmann and van Soest (2001) show that more than half of the within-individual variation in language skills is due to measurement errors. This attentuation bias reduces the OLS estimate of language skill on labor market outcomes toward zero (Dustmann and van Soest, 2001, 2002; Isphording and Otten, 2014). Second, there might be unobserved heterogeneity affecting German language proficiency, preferred job search method and labor market outcomes such as ability, motivation and composition of social networks. These omitted variables result in an upward bias of the language skill estimate. Therefore, the direction of bias of our estimates is \textit{a priori} not clear in this context.

Third, the implementation of a mediation analysis relies on two sequential ignorability assumptions (Imai et. al., 2010). First, given the observed pre-treatment confounders, the treatment assignment is assumed to be ignorable. As an observational study, this part of the assumption is not satisfied because individuals may self select into different levels of language proficiency. Second, the mediator is assumed ignorable given the observed treatment and pre-treatment confounders. This part of assumption may also not hold because the use of social network to search for job is not randomly assigned by the researcher.

I propose several strategies to mitigate these problems. First, I include a full set of control variables to appropriately adjust for the observed difference between treatment and control groups \footnote {\tiny {The treatment variable in this study is the level of German language proficiency, which is a continuous variable. So this sentence means that we have various groups with different levels of treatment.}} to make the ignorability of treatment assignment more credible. This practice, unfortunately, does not rule out the possibility that the model still contains omitted variable bias. Second, I perform the sensitivity analysis to see how robust the empirical findings are to the violation of the sequential ignorability assumption. Third and most importantly, I interact the linguistic distance between German and the respective national language of an immigrant with his/her age at migration to create a proxy for his/her pre-migration German language proficiency. These two variables are strong determinants of language skills (Bleakly and Chin, 2004; Isphording, 2013). As the linguistic distance and the age at migration are plausibly exogenous, this practice helps alleviate the problem of the non-randomness of self-assessed language skills.

The empirical analysis utilizes a unique dataset on newly arrived immigrants to Germany, the IAB-SOEP Migration Sample (2013 - 2016) and has the following findings. First, better pre--migration language skills reduce the probability of finding jobs using social networks by 10 percent. Second, using social networks to find jobs associates with roughly 7 percentage points lower wages, and 3 percent lower probability of working in jobs with highly complex tasks, probably due to the bad quality of the social network composition. Third, direct effect of pre-migration language skills on labor market outcomes accounts for roughly 85 percent, and the mediating effect of job search accounts for around 15 percent. These effect sizes are significant at 5 percent, which points to the importance of applying mediation analysis in understanding the impact of language proficiency on labor market success of immigrants.

This research contributes to the existing literature in several ways. First, by focusing on the pre--migration language skills on the contemporary labor market performance of immigrants in Germany, we have an insight into the role of human capital formation in source countries. Second, using German context where there is a large diversity of immigrants from different linguistic backgrounds (Isphording and Otten, 2014), this paper attempts to open the black-box and explain why and how pre-migration language skills can have such an impact on subsequent labor market outcomes. Third, using the linguistic distance drawn from the Automatic Similarity Judgement Program (ASJP) to construct the proxy variable for language proficiency could mitigate the measurement error problem in self-assessed language skills.

The remainder of the paper is organized as follows: Section 2 presents individual and country-level data. Section 3 formulates the estimation models. Section 4 reports and analyzes the impacts of pre--migration language skills on early job search and contemporary labor market outcomes. Finally, section 5 gives concluding remarks and proposes policy recommendations to improve the labor market success of immigrants in Germany.

\section{Data}
\subsection{Sample Restrictions}

The IAB--SOEP Migration Sample is a household survey undertaken annually by the Institute for Employment Research (IAB) in Nuremberg and the German Socio-Economic Panel (SOEP) at German Institute for Economic Research (DIW Berlin). The first wave of the survey started in 2013 and covered 4,964 persons in 2,723 households. These are second-generation immigrants and immigrants who first appeared in administrative data from the German unemployment insurance in 1995 and are older than 16 years. Some countries of origin are given higher sampling probability to ensure sufficient observations such as immigrants from EU-New Member States and Southern European countries (Bruecker et al., 2014). This new dataset provides a large representative sample of migrants in Germany. With comprehensive information, it also provides new insights to various aspect of immigration such as migration history, education background and labor market performance (Bruecker et al., 2014).

This paper only focus on immigrants who are from 18 to 64 years of age who are employed on full--time or part--time basis. I exclude immigrants in education, retirement, civil or military service and self-employment because of their irregular employment activities and unreliable wage information. Further, I restrict the sample to immigrants who first migrated to Germany when they are at least 18 years of age, and did not conduct study or vocational training. This restriction is necessary to omit immigrants who are born in Germany, or migrated to Germany as children because the language fluency at the time of migration and at the time of job search could be very different. More importantly, to explore the mediating role of using social networks to find jobs on contemporary labor market outcomes, I need the information of job search channels for the current jobs of all individuals. Unfortunately, the IAB-SOEP Migration Sample only provides information about the job search channels of the very first jobs obtained after migration (Dustmann et al., 2016). This information limits the ability to consider immigrants who change their occupations. Therefore, I only consider immigrants who stay in their first jobs after migration so that I am able to follow them from the beginning of searching for jobs until working in the current jobs. 

The sample of analysis consists of xxx immigrants. Labor, family, refugees, regions of origin....%

\subsection{Variables}
\textbf{Dependent Variables} \\
I consider two labor market outcome variables in this research. First, the natural logarithm of hourly wages measures the financial rewards to language fluency and the use of social networks to find jobs. I convert weekly wages to hourly wages based on the actual number of working hours per week, and deflate the results to the base year in 2010. Second, the level of job complexity is used as a proxy for the quality of jobs. In particular, I extract the requirement levels of jobs according to the German Classification of Occupations 2010 (\textit{Klassification der Berufe - KldB 2010}). There are four levels of job requirement as follows: (1) Unskilled or semi-skilled activities (no vocational qualification required), (2) Specialist activities (at least two years of vocational training), (3) Complex specialist activities (master craftsman or technician with technical school qualification), and (4) Highly complex activities (completed university studies of at least four years). \\

\textbf{Mediator Variable}
The mediator variable in this research is the use of social network to obtain the first job after migration to Germany. As only immigrants who stay in their first jobs after migration remain in our sample, this is also the job search method for these immigrants' current jobs. The IAB-SOEP Migration sample includes a number of choices to search for jobs: (1) public employment services, (2) private employment services, (3) job advertisements on public media and (4) personal social networks (friends, family, and business partners from home countries). I generate a dummy variable for using social networks to obtain current jobs which equals to 1 if Yes and 0 otherwise. One immigrant can use multiple job search channels to search for jobs, however, there should be only one realized method that results in the current job. I consider only personal social network because in our sample, more than 65 percent of immigrants indicate that this is their channel of getting current jobs. Due to the limited number of observations, generating a dichotomy indicating the use of social networks to obtain current jobs is better than considering several different job search methods. \\

\textbf{Independent Variables}

\textbf{Control Variables}

\begin{center}
[INSERT TABLE 1 ABOUT HERE]
\end{center}

Table 1 presents summary statistics for two samples: the wage sample and the level of job complexity sample, due to the difference in the number of observations between these two samples. However, the statistics of all covariates in both samples are largely the same.  There are two measures of pre-migration language skills: the ordinal variable measures the the language fluency in the 1 - 5 scale (1: Not at all, 2: Poor, 3: Fair, 4: Good, 5: Very good), and the continuous variable as a result of the principle component analysis to extract information from three skills (speaking, writing and reading). The language skills at migration of immigrants in Germany are rather low and concentrated in the lower bound of the scale.

The average age and age at migration of immigrants is quite high (40 and 29 years old, respectively). More than half of these immigrants have only foreign degrees and were employed before they came to Germany, this reflects our sample restriction to immigrants who did not come to Germany to study. In addition, the majority of immigrants works in medium and large establishment with up to 2000 workers (66 percent). They work mostly in Service sector with approximately 60 percent. The highest proportion of immigrants comes from Eastern European countries.

I employ the linguistic distance between German and the respective national language of an immigrant, and interact with his/her age at migration to create a proxy variable for the language proficiency of an immigrants. The idea behind this approach is based on the fact that the efficiency of learning a language depends on \textit{how far or close} a person' mother tongue to a foreign language that the person wants to learn, and on \textit{how young or old} a person starts to be exposed to the language. The Max Planck Institute of Evolutionary Anthropology has quantified this distance based on the pronunciation of 40 most frequently used words in each language in the so--called Swadesh list. In this paper, I use the results from their Automatic Similarity Judgement Program (ASJP, version 2016). The result is presented in Figure 1. Overall, languages in East Asian countries such as China, Korea and Japan have the longest distance to German, and languages from neighboring countries such as the Netherlands, Switzerland and Luxembourg have the closest distance.

\section{Empirical strategies}

This section presents the methodology to investigate the impact of pre-migration language skills of immigrants on (1) using social network to find the current jobs and (2) contemporary labor market success. First, consider the following set of equations:

$$Y_{i} = \alpha_{1} + \beta_{1}L_{i} + \xi_{1}X_{i} + \epsilon_{i1} (1)$$
$$M_{i} = \alpha_{2} + \beta_{2}L_{i} + \xi_{2}X_{i} + \epsilon_{i2} (2) $$
$$Y_{i} = \alpha_{3} + \beta_{3}L_{i} + \gamma M_{i} + \xi_{3}X_{i} + \epsilon_{i3} (3)$$

in which $Y_{i}$ is the labor market outcome which can be either natural logarithm of hourly wages or the level of job complexity, $L_{i}$ is the continuous measure of German language skills at migration, $M_{i}$ is the binary mediator (the use of social network to find current jobs), $X_{i}$ contains all control variables, $\epsilon_{i}$ is the idiosyncratic error terms. Traditionally, under the LSEM framework, $\hat \beta_{3}$ measures the direct effect of language skills on labor market outcomes, and $\hat \beta_{2}\hat \gamma$ measures the mediation effect of using social network to search for jobs on labor market outcomes. The total effect $\hat \beta_{1}$ equals the sum of $\hat \beta_{3} + \hat \beta_{2}\hat \gamma$, so equation 1 is redundant. 

As long as the linearity assumption holds, under the sequential ignorability and no-interaction assumption, the estimate of mediation effect based on the product of coefficients method is asymptotically consistent. However, as I have a binary mediator (using social networks to search for jobs or not) and an ordered outcome (level of job complexity in the scale from 1 to 4), the linearity assumption does not hold, and we have to apply another approach to accommodate for these non-linear relationships. This paper uses the potential outcomes framework in the spirit of Imai et al. (2010). In particular, I use probit model for equation (2) and linear regression for wages or ordered probit model for the level of job complexity in equation (3).

The confidence interval for the mediation, direct and total effect is estimated using the nonparametric bootstrapping method.

\section{Findings and Discussion}
This section reports the findings from the mediation analysis. First, I show the relationship between language proficiency and the use of social networks to find jobs. Second, I show the relationship between language proficiency and the use of social networks to find jobs on two outcome variables, namely, the hourly earnings and the level of job complexity.

\begin{center}
[INSERT TABLE 2 ABOUT HERE]
\end{center}

Table 2 reports the impacts of pre-migration language skills on the use of social networks to find jobs in Germany. Better pre--migration language skills reduce the probability of finding jobs using social networks by 10 percent. Additionally, Table 2 presents some interesting findings about other aspects of human capital. For instance, immigrants who have higher qualifications have significantly less probability to find job via informal networks as they are more likely to have access to formal information channels such as job advertisements on newspapers and the Internet. Immigrants who were previously employed in their home countries are also less likely to find job via social networks. On the contrary, immigrants who have foreign degrees are more likely to leverage on their social networks to find their first jobs in Germany.

\begin{center}
[INSERT TABLE 3 ABOUT HERE]
\end{center}

I move to analyze the impact of German language skills upon arrival on hourly wages with the mediating effect of using social networks to find jobs in Table 3 and on the level of job complexity in Table 4. Table 3 shows that using social networks to find jobs associates with roughly 7 percentage points lower wages.

\begin{center}
[INSERT TABLE 4 ABOUT HERE]
\end{center}
Table 4 shows that sing social networks to find jobs associates with 3 percent lower probability of working in jobs with highly complex tasks, probably due to the bad quality of the social network composition. 

\begin{center}
[INSERT FIGURE 4 AND 5 ABOUT HERE]
\end{center}

Direct effect of pre-migration language skills on labor market outcomes accounts for roughly 85 percent, and the mediating effect of job search accounts for around 15 percent. These effect sizes are significant at 5 percent, which points to the importance of applying mediation analysis in understanding the impact of language proficiency on labor market success of immigrants. 

All in all, pre--migration language skills are particularly important to understand the job search behavior of immigrants in Germany. In addition, pre--migration language skills are also good predictors for current labor market outcomes such as hourly earnings, and level of job complexity.

\begin{center}
[to be continued]
\end{center} 

\section{Conclusions}

Germany has a long tradition of immigration among European countries, but unlike the United States, Australia and Canada, policy makers in Germany are not well-prepared for the integration of the existing population of immigrants (Dustmann, Frattini and Lanzara, 2012). This paper contributes to the existing literature by investigating the effects of language skills upon arrival on early job search methods and contemporary labor market outcomes of immigrants in Germany. Good pre-migration language skills reduce the probability of using informal social networks to search for jobs, and increase access to other formal channels such as public/private employment services or job advertisement on mass media. Finding jobs via social networks is also shown to have negative impact on earnings and level of job complexity.

The paper sheds light on a potential mechanism through which pre-migration language skills have an impact on current labor market performance. Thereby, the paper contributes to both strands of literature in labor market integration of immigrants: (1) job search behavior of immigrants and (2) the impact of job search channels on labor market performance. Understanding the job search process and the determinants for successful labor market integration is the key to design effective policies to improve labor market perspectives of immigrants in Germany.

\section{Appendix}
\begin{center}
% Table generated by Excel2LaTeX from sheet 'Summary statistics'
\begin{table}[htbp]
  \centering
  \caption{Summary Statistics}
\scalebox{0.8}{
    \begin{tabular}{lrrrrr}
    \toprule
    \toprule
    \multicolumn{1}{c}{\textbf{Variables}} & \multicolumn{1}{c}{\textbf{Mean}} & \multicolumn{1}{c}{\textbf{Std. Dev}} & \multicolumn{1}{c}{\textbf{Min}} & \multicolumn{1}{c}{\textbf{Max}} & \multicolumn{1}{c}{\textbf{Observations}} \\
    \midrule
    \textbf{Dependent variables} &       &       &       &       &  \\
    Log(hourly wages) & 2.440 & 0.480 & 0.435 & 4.511 & 1,078 \\
    Level of job complexity & 1.984 & 0.975 & 1     & 4     & 1,149 \\
    \textbf{Mediator} &       &       &       &       &  \\
    Social network (1/0) & 0.647 & 0.478 & 0     & 1     & 1,149 \\
    \textbf{Independent variables} &       &       &       &       &  \\
    Pre-migration German language skills (ordinal) & 2.082 & 1.310 & 1     & 5     & 1,149 \\
    Pre-migration German language skills (factor analysis) & 0.000 & 1.000 & -1.066 & 1.474 & 1,149 \\
    Linguistic distance & 93.079 & 7.769 & 0     & 102.510 & 1,149 \\
    Age at migration & 29.066 & 11.270 & 0     & 57 & 1,149 \\
    Log(Linguistic distance * Age at migration) & 7.760 & 0.766 & 0     & 8.639 & 1,149 \\
    \textbf{Control variables} &       &       &       &       &  \\
    Males (1/0) & 0.503 & 0.500 & 0     & 1     & 1,149 \\
    Married (1/0) & 0.720 & 0.449 & 0     & 1     & 1,149 \\
    Age   & 39.286 & 10.541 & 18    & 65    & 1,149 \\
    Years since migration & 10.220 & 7.797 & 0     & 46    & 1,149 \\
    Years of education & 11.274 & 2.345 & 7     & 18    & 1,149 \\
    \multicolumn{1}{p{21.85em}}{Highest qualification (1= No, 2 = Secondary, 3 = Vocation, 4 = University)} & 2.904 & 0.902 & 1     & 4     & 1,149 \\
    Only foreign degree (1/0) & 0.572 & 0.495 & 0     & 1     & 1,149 \\
    Employed before migration (1/0) & 0.612 & 0.488 & 0     & 1     & 1,149 \\
    Tenure & 6.375 & 6.067 & 0     & 42    & 1,149 \\
    Establishment size (1= Large, 0 = Small) & 0.663 & 0.473 & 0     & 1     & 1,149 \\
    \multicolumn{1}{p{21.85em}}{Industry (1= Agriculture, 2 = Manufacturing, 3 = Services)} & 2.488 & 0.696 & 1     & 3     & 1,149 \\
    \multicolumn{1}{p{21.85em}}{Regions of settlement (1= NE, 2 = NW, 3 = S)} & 2.271 & 0.724 & 1     & 3     & 1,149 \\
    Survey years & 2014.129 & 1.129 & 2013  & 2016  & 1,149 \\
    \bottomrule
    \end{tabular}%
 }
 \begin{tablenotes}
      \small
      \item Source: IAB-SOEP Migration Sample (Wave: 2013 - 2016).
    \end{tablenotes}
\end{table}%
\end{center}

% Table generated by Excel2LaTeX from sheet '20180504 Regression'
\begin{table}[htbp]
  \centering
  \caption{Moderation equation - Language skills and the use of Social networks to search for jobs}
\scalebox{0.8}{
   \begin{tabular}{p{15em}rrrrr}
    \toprule
    \toprule
    \multicolumn{1}{r}{} & \multicolumn{2}{p{9.55em}}{\textbf{Self-assessed language skills}} &       & \multicolumn{2}{p{11.55em}}{\textbf{Linguistic distance * Age at Migration}} \\
\cmidrule{2-3}\cmidrule{5-6}    \multicolumn{1}{r}{} & \multicolumn{1}{c}{(1)} & \multicolumn{1}{c}{(2)} &       & \multicolumn{1}{c}{(3)} & \multicolumn{1}{c}{(4)} \\
    \midrule
    \multicolumn{1}{l}{\textit{Dep. Var: Use Social networks to search for jobs (1/0)}} &       &       &       &       &  \\
    Pre-migration language skills & -0.097*** & -0.093*** &       & 0.079*** & 0.074** \\
                              & (0.022) & (0.022) &       & (0.027) & (0.034) \\
    Male  &       & 0.012 &       &       & 0.004 \\
    \multicolumn{1}{r}{} &       & (0.041) &       &       & (0.041) \\
    Age at migration &       & 0.005** &       &       & 0.001 \\
                              &       & (0.002) &       &       & (0.002) \\
    \textbf{Qualification} &       &       &       &       &  \\
    No qualification &       & Ref   &       &       & Ref \\
    Secondary school &       & -0.037 &       &       & -0.064 \\
                              &       & (0.086) &       &       & (0.087) \\
    Vocational training &       & -0.164* &       &       & -0.211** \\
                              &       & (0.093) &       &       & (0.094) \\
    University &       & -0.203** &       &       & -0.268*** \\
                              &       & (0.100) &       &       & (0.099) \\
    Only Foreign degree (Yes) &       & 0.071 &       &       & 0.075 \\
    \multicolumn{1}{r}{} &       & (0.061) &       &       & (0.061) \\
    Employed before migration (Yes) &       & -0.048 &       &       & -0.061 \\
    \multicolumn{1}{r}{} &       & (0.046) &       &       & (0.046) \\
    \textbf{Industry} &       &       &       &       &  \\
    Agriculture &       & Ref   &       &       & Ref \\
    Manufacturing &       & -0.003 &       &       & 0.006 \\
    \multicolumn{1}{r}{} &       & (0.066) &       &       & (0.067) \\
    Services &       & -0.034 &       &       & -0.029 \\
    \multicolumn{1}{r}{} &       & (0.058) &       &       & (0.060) \\
    \multicolumn{1}{l}{Pseudo-R squared} & 0.024 & 0.051 &       & 0.012 & 0.036 \\
    Observations &       1,149    &        1,149    &       &            1,149    &            1,149    \\
    \bottomrule
    \end{tabular}%
}

\begin{tablenotes}
      \small
      \item Source: IAB-SOEP Migration Sample (Wave: 2013 - 2016).
      \item Notes: Dependent variable: Use social networks to find current jobs (1/0). Sample of individuals aged 18 to 65, having full-time or part-time employment, stayed in their first jobs in Germany. Average Marginal Effects after probit regression are reported in this table. Clustered standard errors are reported in parentheses.  ***, ** and * indicate significance at the 1\%, 5\% and 10\% level, respectively. In addition to the covariates shown in the table, models (2) and (4)  include dummies for regions of settlement and dummies for survey years.
    \end{tablenotes}
\end{table}%

% Table generated by Excel2LaTeX from sheet '20180504 Regression'
\begin{table}[htbp]
  \centering
  \caption{Outcome equation - Impact of language skills and social networks as a job search channel on Earnings}
\scalebox{0.8}{
    \begin{tabular}{p{15em}rrrrr}
    \toprule
    \toprule
    \multicolumn{1}{r}{} & \multicolumn{2}{p{9.55em}}{\textbf{Self-assessed language skills}} &       & \multicolumn{2}{p{11.55em}}{\textbf{Linguistic distance * Age at Migration}} \\
\cmidrule{2-3}\cmidrule{5-6}    \multicolumn{1}{r}{} & \multicolumn{1}{c}{(1)} & \multicolumn{1}{c}{(2)} &       & \multicolumn{1}{c}{(3)} & \multicolumn{1}{c}{(4)} \\
    \midrule
    \multicolumn{1}{r}{} &       &       &       &       &  \\
    \multicolumn{1}{l}{Find job by social networks} & -0.075* & -0.064* &       & -0.094** & -0.071* \\
    \multicolumn{1}{l}{                         } & (0.040) & (0.034) &       & (0.043) & (0.041) \\
    Pre-migration language skills & 0.077*** & 0.040** &       & -0.022 & -0.046* \\
    \multicolumn{1}{l}{                         } & (0.023) & (0.019) &       & (0.020) & (0.028) \\
    Male  &       & 0.145*** &       &       & 0.148*** \\
    \multicolumn{1}{l}{                         } &       & (0.036) &       &       & (0.036) \\
    \multicolumn{1}{l}{Age} &       & 0.018* &       &       & 0.023** \\
    \multicolumn{1}{l}{                         } &       & (0.009) &       &       & (0.009) \\
    \multicolumn{1}{l}{Year since migration} &       & -0.012* &       &       & -0.013** \\
    \multicolumn{1}{l}{                         } &       & (0.007) &       &       & (0.006) \\
    \textbf{Qualification} &       &       &       &       &  \\
    No qualification &       & Ref      &       &       & Ref  \\
    Secondary school &       & -0.023 &       &       & -0.011 \\
                              &       & (0.072) &       &       & (0.062) \\
    Vocational training &       & 0.039 &       &       & 0.055 \\
                              &       & (0.069) &       &       & (0.065) \\
    University &       & 0.344*** &       &       & 0.368*** \\
    \multicolumn{1}{l}{                         } &       & (0.078) &       &       & (0.067) \\
    \multicolumn{1}{l}{Tenure} &       & 0.025*** &       &       & 0.024*** \\
    \multicolumn{1}{l}{                         } &       & (0.008) &       &       & (0.007) \\
    \multicolumn{1}{l}{Establishment size (Large)} &       & 0.148*** &       &       & 0.152*** \\
    \multicolumn{1}{l}{                         } &       & (0.032) &       &       & (0.035) \\
    \multicolumn{1}{l}{Constant                 } & 2.517*** & 1.670*** &       & 2.673*** & 1.871*** \\
    \multicolumn{1}{l}{                         } & (0.033) & (0.221) &       & (0.139) & (0.242) \\
    \multicolumn{1}{l}{R-squared} & 0.027 & 0.317 &       & 0.011 & 0.315 \\
    Observations &       1,078    &        1,078    &       &            1,078    &            1,078    \\
    \bottomrule
    \end{tabular}%
}

\begin{tablenotes}
      \small
      \item Source: IAB-SOEP Migration Sample (Wave: 2013 - 2016).
      \item Notes: Dependent variable: Natural logarithm of hourly earnings . Sample of individuals aged 18 to 65, having full-time or part-time employment, stayed in their first jobs in Germany. Clustered standard errors are reported in parentheses.  ***, ** and * indicate significance at the 1\%, 5\% and 10\% level, respectively. In addition to the covariates shown in the table, models (2) and (4) include age squared, years since migration squared, tenure squared, dummies for industries, dummies for regions of settlement and dummies for survey years.
    \end{tablenotes}
\end{table}%

\begin{landscape}
% Table generated by Excel2LaTeX from sheet '20180504 Regression'
\begin{table}[htbp]
  \centering
  \caption{Outcome equation - Level of job complexity}
    \begin{tabular}{p{15em}rrrrrrrrr}
    \toprule
    \toprule
    \multicolumn{1}{r}{} & \multicolumn{4}{c}{\textbf{Self-assessed language skills}} &       & \multicolumn{4}{c}{\textbf{Linguistic distance * Age at Migration}} \\
\cmidrule{2-5}\cmidrule{7-10}    \multicolumn{1}{r}{} & \multicolumn{1}{c}{\textit{Level 1}} & \multicolumn{1}{c}{\textit{Level 2}} & \multicolumn{1}{c}{\textit{Level 3}} & \multicolumn{1}{c}{\textit{Level 4}} &       & \multicolumn{1}{c}{\textit{Level 1}} & \multicolumn{1}{c}{\textit{Level 2}} & \multicolumn{1}{c}{\textit{Level 3}} & \multicolumn{1}{c}{\textit{Level 4}} \\
    \midrule
    \multicolumn{1}{l}{Find job by social networks} & 0 .054** & -0.016** & -0.007* & -0.031** &       &  0.060* &  -0.017* &  -0.008 &  -0.034* \\
    \multicolumn{1}{l}{                         } & (0.027) & (0.008) & (0.004) & (0.015) &       & (0.035) & (0.009) & (0.005) & (0.021) \\
    Pre-migration language skills & -0.040** & 0.012** & 0.005** & 0.023** &       &  0.057*** &  -0.017*** &  -0.008*** &  -0.033*** \\
    \multicolumn{1}{l}{                         } & (0.017) & (0.005) & (0.002) & (0.009) &       & (0.017) & (0.005) & (0.002) & (0.009) \\
    \multicolumn{1}{l}{Pseudo-R-squared} & 0.203 & 0.203 & 0.203 & 0.203 &       & 0.203 & 0.203 & 0.203 & 0.203 \\
    Observations &        1,149    &        1,149    &        1,149    &        1,149    &       &         1,149    &         1,149    &         1,149    &         1,149    \\
    \bottomrule
    \end{tabular}%
\begin{tablenotes}
      \small
      \item Source: IAB-SOEP Migration Sample (Wave: 2013 - 2016).
      \item Notes: Dependent variable: Level of Job complexity (1/4) . Sample of individuals aged 18 to 65, having full-time or part-time employment, stayed in their first jobs in Germany. Clustered standard errors are reported in parentheses.  ***, ** and * indicate significance at the 1\%, 5\% and 10\% level, respectively. In addition to the covariates shown in the table, all models include dummy for males, age, age squared, years since migration, years since migration squared, dummies for highest qualifications, tenure squared, dummy for establishment size, dummies for industries, dummies for regions of settlement and dummies for survey years.
    \end{tablenotes}
\end{table}%

\end{landscape}

\begin{figure}
\centering     %%% not \center
\includegraphics[scale=1.2]{Figure7}
\caption{Linguistic distance between German and other languages}
\end{figure}

\begin{figure}
\centering     %%% not \center
\includegraphics[scale=0.8]{Figure1}
\caption{Linguistic distance, Language skills and the use of Social Networks to find jobs}
\end{figure}

\begin{figure}
\centering     %%% not \center
\includegraphics[scale=0.8]{Figure2}
\caption{Language skills, the use of Social Networks to find jobs and Wages}
\end{figure}

\begin{figure}
\centering     %%% not \center
\includegraphics[scale=1.2]{Figure3}
\caption{Total effects of language skills on wages decomposed}
\end{figure}

\begin{figure}
\centering     %%% not \center
\includegraphics[scale=1.2]{Figure4}
\caption{Total effects of language skills on level of job complexity decomposed}
\end{figure}

\begin{figure}
\centering     %%% not \center
\includegraphics[scale=1.2]{Figure5}
\caption{Sensitivity analysis - Outcome: log(hourly wages)}
\end{figure}

\begin{figure}
\centering     %%% not \center
\includegraphics[scale=1.2]{Figure6}
\caption{Sensitivity analysis - Outcome: Level of job complexity}
\end{figure}

\end{document}



