%%%%%%%%%%%%%%%%%%%%%%%%%%%%%%%%%%%%%%%%%
% Beamer Presentation
% LaTeX Template
% Version 1.0 (10/11/12)
%
% This template has been downloaded from:
% http://www.LaTeXTemplates.com
%
% License:
% CC BY-NC-SA 3.0 (http://creativecommons.org/licenses/by-nc-sa/3.0/)
%
%%%%%%%%%%%%%%%%%%%%%%%%%%%%%%%%%%%%%%%%%

%----------------------------------------------------------------------------------------
%	PACKAGES AND THEMES
%----------------------------------------------------------------------------------------

\documentclass{beamer}

\mode<presentation> {

% The Beamer class comes with a number of default slide themes
% which change the colors and layouts of slides. Below this is a list
% of all the themes, uncomment each in turn to see what they look like.

%\usetheme{default}
%\usetheme{AnnArbor}
%\usetheme{Antibes}
%\usetheme{Bergen}
%\usetheme{Berkeley}
%\usetheme{Berlin}
%\usetheme{Boadilla}
%\usetheme{CambridgeUS}
%\usetheme{Copenhagen}
%\usetheme{Darmstadt}
%\usetheme{Dresden}
%\usetheme{Frankfurt}
%\usetheme{Goettingen}
%\usetheme{Hannover}
%\usetheme{Ilmenau}
%\usetheme{JuanLesPins}
%\usetheme{Luebeck}
\usetheme{Madrid}
%\usetheme{Malmoe}
%\usetheme{Marburg}
%\usetheme{Montpellier}
%\usetheme{PaloAlto}
%\usetheme{Pittsburgh}
%\usetheme{Rochester}
%\usetheme{Singapore}
%\usetheme{Szeged}
%\usetheme{Warsaw}

% As well as themes, the Beamer class has a number of color themes
% for any slide theme. Uncomment each of these in turn to see how it
% changes the colors of your current slide theme.

%\usecolortheme{albatross}
%\usecolortheme{beaver}
%\usecolortheme{beetle}
%\usecolortheme{crane}
%\usecolortheme{dolphin}
%\usecolortheme{dove}
%\usecolortheme{fly}
%\usecolortheme{lily}
%\usecolortheme{orchid}
%\usecolortheme{rose}
%\usecolortheme{seagull}
%\usecolortheme{seahorse}
%\usecolortheme{whale}
%\usecolortheme{wolverine}

%\setbeamertemplate{footline} % To remove the footer line in all slides uncomment this line
%\setbeamertemplate{footline}[page number] % To replace the footer line in all slides with a simple slide count uncomment this line

%\setbeamertemplate{navigation symbols}{} % To remove the navigation symbols from the bottom of all slides uncomment this line
}

\usepackage{graphicx} % Allows including images
\usepackage{booktabs} % Allows the use of \toprule, \midrule and \bottomrule in tables
\usepackage{color,hyperref}
\definecolor{darkblue}{rgb}{0.0,0.0,0.3}
\hypersetup{colorlinks,breaklinks,
            linkcolor=darkblue,urlcolor=darkblue,
            anchorcolor=darkblue,citecolor=darkblue}
\usepackage{booktabs, multicol, multirow}
\usepackage{epstopdf}
\usepackage{subfigure}
\usepackage{xspace}
\usepackage{ltablex}
\usepackage[flushleft]{threeparttable}
\usepackage{adjustbox}
\usepackage{pdflscape}
\usepackage{tabu}
\usepackage{caption}


%----------------------------------------------------------------------------------------
%	TITLE PAGE
%----------------------------------------------------------------------------------------

\title[]{Coming to Germany: Pre-migration language skills, job search and labor market outcomes} % The short title appears at the bottom of every slide, the full title is only on the title page

\author{Huy Le Quang} % Your name
%\institute[] % Your institution as it will appear on the bottom of every slide, may be shorthand to save space
%{University of Bamberg\\ % Your institution for the title page
%\medskip
%\large{BGPE Research Workshop}} % Your email address

\date{07.06.2018} % Date, can be changed to a custom date

\begin{document}

\begin{frame}
\titlepage % Print the title page as the first slide
\end{frame}

\begin{frame}
\frametitle{Overview} % Table of contents slide, comment this block out to remove it
\tableofcontents % Throughout your presentation, if you choose to use \section{} and \subsection{} commands, these will automatically be printed on this slide as an overview of your presentation
\end{frame}

%----------------------------------------------------------------------------------------
%	PRESENTATION SLIDES
%----------------------------------------------------------------------------------------

%------------------------------------------------
\section{Motivation} % Sections can be created in order to organize your presentation into discrete blocks, all sections and subsections are automatically printed in the table of contents as an overview of the talk
%------------------------------------------------

\begin{frame}
\frametitle{Why focus on pre-migration language skills?}
\begin{itemize}

\item 8/2007: To promote integration ability, Germany applied pre-departure requirements (\textit{formal language certificates}) to all third-country nationals coming for family reasons  

$\rightarrow$ Do people with some host-country language skills at arrival integrate better in the long-run than the ones who do not? \\
$\rightarrow$ Should the admission of family migrants not only be grounded on the moral aspect of family life for which their language skills have no intrinsic value?

\item Methodological issues in previous literature:

$\rightarrow$ Focus on current language skills $\rightarrow$ regression could be spurious (\textit{due to reverse causality}) \\
$\rightarrow$ Focus on total effect $\rightarrow$  not empirically clear (1) why such an effect exist, (2) size of direct/indirect effects
\end{itemize}

\end{frame}

%------------------------------------------------

\begin{frame}
\frametitle{What if you have better language skills at entry?}

\begin{itemize}
\item Signal a readily available human capital component and willingess to integrate
\item Complement other components of human capital earned in source countries 
\item Choose an occupation with more interactive tasks, and eventually earn more than the ones working with manual tasks
\item Have better access to formal job vacancy information in public media or employment agencies
\item ...
\end{itemize}

\end{frame}

%------------------------------------------------
\begin{frame}
\frametitle{Job search behavior of immigrants}
50 - 60\% immigrants obtain jobs through social networks, because of:

\begin{itemize}
\item Limited language skills (\textit{DIW, 2016; Battu et al., 2011})
\item Low level of assimilation (\textit{Battu et al., 2011})
\item Limited knowledge of local labor market institutions (\textit{Mahuteau and Junankar, 2007})
\end{itemize}

\textbf{But} immigrants tend to have small and homogenous social network \\
$\rightarrow$ Narrow range of potential job opportunities, increased duration of job search (\textit{Beggs and Chapman, 1990}) \\
$\rightarrow$ Lower wages, higher mismatch and job dissatisfaction (\textit{Carlsson et al., 2014; Battu et al., 2011; Lancee, 2010})

\end{frame}

%------------------------------------------------

\begin{frame}
\frametitle{Research objectives}
\textbf {Research question:}
\begin{itemize}
\item Does pre-migration language skills affect current labor market outcomes of immigrants in Germany? And how?
\end{itemize}

\textbf {Contributions:}
\begin{itemize}
\item Propose a mechanism through which \textit{pre-migration} language skills exert the influence on \textit{current} labor market outcomes
\item Apply non-traditional mediation analysis with less assumptions 
\end{itemize}
\end{frame}

%------------------------------------------------
\begin{frame}

\frametitle{Preliminary results}
\begin{itemize}
\item Using social networks to obtain jobs is a popular method among immigrants with bad language skills
\item But using social network to obtain jobs associates with lower earnings and lower level of job complexity
\item No significant evidence for family migrants (\textit{even they are the main subject of the pre-departure integration requirement})
\end{itemize}

\end{frame}



%------------------------------------------------
\section{Methodology}
%------------------------------------------------

\begin{frame}
\frametitle{Mediation analysis}
\begin{figure}[ht]
        \begin{minipage}[b]{0.45\linewidth}
            \centering
            \includegraphics[width=\textwidth]{simple.png}
            \caption{Simple mediation}
            
        \end{minipage}
        \hspace{0.5cm}
        \begin{minipage}[b]{0.45\linewidth}
            \centering
            \includegraphics[width=\textwidth]{moderated.png}
            \caption{Moderated mediation}
            
        \end{minipage}
    \end{figure}

\end{frame}

%------------------------------------------------

\begin{frame}
\frametitle{Mediation analysis - Potential outcome framework (1)}
\begin{enumerate}

\item Language skills: a latent continuous variable: $l_{0}, l_{1}$... \\

\item Assume two choices of obtaining jobs
\begin{itemize}
\item M = 1 (Use social networks)
\item M = 0 (otherwise)
\end{itemize}

\item Labor market outcomes (Y)
\begin{itemize}
\item Hourly earnings (continuous Var.)
\item Level of job complexity (ordered categorical Var.)
\end{itemize}

\end{enumerate}

\end{frame}

%------------------------------------------------

\begin{frame} % Need to use the fragile option when verbatim is used in the slide
\frametitle{Mediation analysis - Potential outcome framework (2)}
\textbf{Total (unit) effect}
$$\tau_{i} = Y_{i}(l_{1}, M_{i}(l_{1})) - Y_{i}(l_{0}, M_{i}(l_{0}))$$

\textbf{Mediation effect}
$$\delta_{i} = Y_{i}(l, M_{i}(l_{1})) - Y_{i}(l, M_{i}(l_{0}))$$

\textbf{Direct effect}
$$\xi_{i} = Y_{i}(l_{1}, M_{i}(l)) - Y_{i}(l_{0}, M_{i}(l))$$

\textbf{Assumption}: Sequential Ignorability Assumption
$$\left \{Y_{i}(l', m), M_{i}(l)\right \} \perp L_{i}|X_{i}=x, $$
$$Y_{i}(l', m) \perp M_{i}(l)|L_{i}=l, X_{i}=x $$

\end{frame}

%------------------------------------------------

\begin{frame}
\frametitle{Mediation analysis - Procedure}
\begin{figure}
\centering
\includegraphics[scale=0.6]{procedure.png}
\caption{Procedure to conduct mediation analysis}
\end{figure}
\end{frame}


%------------------------------------------------
\section{Data}
%------------------------------------------------

\begin{frame}
\frametitle{Data and sample restrictions}
\textbf{IAB--SOEP Migration Sample}, 2013 - 2016
\begin{itemize}
\item A survey with sample drawn from IAB Integrated Employment Biographies
\end{itemize}
\textbf{Sample restrictions}
\begin{itemize}
\item Age: 20--64, migration background, full-time or part-time employment
\item Age at migration $\geq$ 18
\item Not migrated to Germany to study/conduct vocational training
\item Stay in the first job after migration ($\sim$ 50\% of sample)
\end{itemize}
$\rightarrow$ Effective sample size: 1,008

\end{frame}

%------------------------------------------------

\begin{frame}
\frametitle{Data - Linguistic Distance}
\begin{itemize}
\item Measure linguistic distance between German and other languages based on difference in pronunciation
\item Data from Automatic Similarity Judgement Program (\textit{Max Planck Institute for Evolutionary Anthropology})
\item Method: Lexicostatistical approach
\end{itemize}

\begin{figure}
\centering
\includegraphics[scale=0.8]{LD.png}
\end{figure}
\end{frame}

%------------------------------------------------
\begin{frame}
\frametitle{Variables}

\textbf{Dependent Variables}: Labor market outcomes
\begin{itemize}
\item Hourly earnings
\item Level of job complexity (1/4), based on KldB--2010
\end{itemize}

\textbf{Independent Variable}: Pre-migration language skills
\begin{itemize}
\item Self-assessed language skills (Speaking, Reading, Writing)
\item PCA to predict latent variable "Language skills": $l_{i} \in L \sim N(0,1)$
\end{itemize}

\textbf{Control Variables}
\begin{itemize}
\item Demographics: age, gender, migration status, migration period, country of origin FEs, Federal States FEs, survey year FEs
\item Human capital: highest qualifications, tenure, employed before migration
\item Firm characteristics: establishment size, industry FEs
\item Moderation: linguistic distance
\end{itemize}

\end{frame}

%---------------------------------------------------------
\section{Descriptive statistics}
%---------------------------------------------------------

\begin{frame}
\frametitle{Using social networks to obtain jobs}

\begin{figure}
\centering
\includegraphics[scale = 0.7]{SNpercent.png}
  \caption{Percentage of using social networks to obtain jobs, by status at entry}
\end{figure}

\end{frame}

%---------------------------------------------------------

\begin{frame}
\frametitle{Language skills and earnings}

\begin{figure}
\centering
\includegraphics[scale = 0.7]{LS_earnings.png}
  \caption*{Better language skills associates with higher earnings}
\end{figure}

\end{frame}

%---------------------------------------------------------

\begin{frame}
\frametitle{Language skills and level of job complexity}

\begin{figure}
\centering
\includegraphics[scale = 0.7]{LS_level.png}
  \caption*{Better language skills associates with higher levels of job complexity}
\end{figure}

\end{frame}

%---------------------------------------------------------

\begin{frame}
\frametitle{Language skills and use social networks to obtain jobs}

\begin{figure}
\centering
\includegraphics[scale = 0.7]{LS_SN.png}
  \caption*{Better language skills associates with lower probability of using social networks to obtain jobs}
\end{figure}

\end{frame}

%---------------------------------------------------------

\begin{frame}
\frametitle{Use social networks to obtain jobs and earnings}

\begin{figure}
\centering
\includegraphics[scale = 0.7]{SN_earnings.png}
  \caption*{Using social networks to obtain jobs associates with lower earnings}
\end{figure}

\end{frame}

%---------------------------------------------------------

\begin{frame}
\frametitle{Use social networks to obtain jobs and level of job complexity}

\begin{figure}
\centering
\includegraphics[scale = 0.7]{SN_level.png}
  \caption*{Using social networks to obtain jobs associates with lower level of job complexity}
\end{figure}

\end{frame}
%--------------------------------------------------------
\section{Empirical results}
%---------------------------------------------------------

\begin{frame}
\frametitle{Empirical result - Earnings}

\begin{table}[htbp]
 \centering
\scalebox{0.60}{
    \begin{tabular}{lrr}

    \toprule
          & \multicolumn{1}{l}{\textbf{Simple mediation}} & \multicolumn{1}{l}{\textbf{Moderated mediation}} \\
    \midrule
    \textbf{Use of social networks to obtain jobs} &       &  \\
    Pre-migration language skills & -0.085*** & -0.083*** \\
          & (0.020) & (0.020) \\
    Linguistic distance &       & 0.055** \\
          &       & (0.024) \\
    \midrule
    \textbf{Earnings} &       &  \\
    Use of social networks to obtain jobs & -0.099*** & -0.063*** \\
          & (0.037) & (0.037) \\
    Pre-migration language skills & 0.039** & 0.036** \\
          & (0.017) & (0.017) \\
    Linguistic distance &       & -0.050 \\
          &       & (0.036) \\
    Social networks * Linguistic distance &       & -0.032 \\
          &       & (0.072) \\
    Language skills * Linguistic distance &       & 0.015 \\
          &       & (0.016) \\
    \midrule
    R-squared & 0.481 & 0.389 \\
    Observations & 965   & 965 \\
    \bottomrule
    \end{tabular}%
}
\begin{tablenotes}
      \tiny
      \item Source: IAB-SOEP Migration Sample (Wave: 2013 - 2016).
      \item Notes: Average Marginal Effects. Robust standard errors are reported in parentheses.  ***, ** and * indicate significance at the 1\%, 5\% and 10\% level, respectively. Models include age, migration status, migration period, tenure, industry FEs, Federal States FEs, survey years FEs. Simple mediation model includes country of origins FEs.
   
 \end{tablenotes}
  
\end{table}%


\end{frame}

%---------------------------------------------------------

\begin{frame}
\frametitle{Empirical result - Level of job complexity}

% Table generated by Excel2LaTeX from sheet 'Tabelle1'

\begin{table}[htbp]
\centering
 \scalebox{0.50}{ 
    \begin{tabular}{lrrrrrrrr}
    \toprule
          & \multicolumn{4}{c}{\textbf{Simple mediation}} & \multicolumn{4}{c}{\textbf{Moderated mediation}} \\
    \midrule
          & \multicolumn{1}{c}{\textit{Level 1}} & \multicolumn{1}{c}{\textit{Level 2}} & \multicolumn{1}{c}{\textit{Level 3}} & \multicolumn{1}{c}{\textit{Level 4}} & \multicolumn{1}{c}{\textit{Level 1}} & \multicolumn{1}{c}{\textit{Level 2}} & \multicolumn{1}{c}{\textit{Level 3}} & \multicolumn{1}{c}{\textit{Level 4}} \\
    Use of social networks to obtain jobs & 0.073** & -0.026** & -0.009** & -0.038** & 0.061** & -0.019** & -0.006* & -0.036** \\
          & (0.030) & (0.011) & (0.004) & -0.016 & (0.030) & (0.009) & (0.003) & (0.018) \\
    Pre-migration language skills & -0.051*** & 0.019*** & 0.006*** & 0.026*** & -0.050*** & 0.016*** & 0.005*** & 0.029*** \\
          & (0.014) & (0.005) & (0.002) & (0.007) & (0.014) & (0.005) & (0.001) & (0.008) \\
    Linguistic distance &       &       &       &       & 0.040* & -0.012 & -0.004* & -0.024** \\
          &       &       &       &       & (0.021) & (0.008) & (0.002) & (0.011) \\
    \midrule
    R-squared & 0.300 & 0.300 & 0.300 & 0.300 & 0.220 & 0.220 & 0.220 & 0.220 \\
    Observations &            1,008    &            1,008    &            1,008    &            1,008    &            1,008    &            1,008    &            1,008    &            1,008    \\
    \bottomrule
    \end{tabular}%
}

 \begin{tablenotes}
      \tiny
      \item Source: IAB-SOEP Migration Sample (Wave: 2013 - 2016).
      \item Notes: Average Marginal Effects. Robust standard errors are reported in parentheses.  ***, ** and * indicate significance at the 1\%, 5\% and 10\% level, respectively. Models include age, migration status, migration period, tenure, industry FEs, Federal States FEs, survey years FEs. Simple mediation model includes country of origins FEs.
   
 \end{tablenotes}

\end{table}%

\end{frame}

%--------------------------------------------------------
\begin{frame}
\frametitle{Effect size - Earnings}
\begin{figure}
\centering
\includegraphics[scale = 0.4]{size_earnings.png}
  \caption{Effect size - Outcome: Earnings}
\caption*{Note: Non-parametric bootstrap 95\%CIs, Simulations = 1,000}
\end{figure}
\end{frame}

%-------------------------------------------------------

\begin{frame}
\frametitle{Effect size - Level of job complexity}

\begin{figure}
\centering
\includegraphics[scale = 0.4]{size_level.png}
  \caption{Effect size - Outcome: Level of job complexity}
\caption*{\small{Note: Non-parametric bootstrap 95\% CIs, Simulations = 1,000}}
\end{figure}
\end{frame}

%-----------------------------------------------------
\begin{frame}
\frametitle{Heterogeneous effects}

% Table generated by Excel2LaTeX from sheet 'Tabelle1'
\begin{table}[htbp]
  \centering
\scalebox{0.76}{
    \begin{tabular}{lrr}
    \toprule
    \multicolumn{1}{c}{\textbf{Status at entry}} & \multicolumn{1}{c}{\textbf{Frequency}} & \multicolumn{1}{c}{\textbf{Percentage}} \\
    \midrule
    Family reasons & 592   & 0.587 \\
    Labor & 308   & 0.306 \\
    Humanitarian reasons & 108   & 0.107 \\
    Total & 1,008 & 1 \\
    \bottomrule
    \end{tabular}%
}
\end{table}%


\begin{itemize}
\item Significant evidence only for Labor migrants
\item No significant evidence for Family migrants and migrants of humanitarian reasons
\end{itemize}

\textbf{Possible explanations:}
\begin{itemize}
\item Language skills required by the Residence Act is only at the basic level (A1)
\item Job search duration of family migrants and migrants of humanitarian reasons are longer than of labor migrants \\ $\rightarrow$ Pre-migration language skills different from skills at the time of starting the job
\end{itemize}

\end{frame}

%-----------------------------------------------------
\section{Conclusion}
%-----------------------------------------------------

\begin{frame}
\frametitle{Conclusions}
\begin{itemize}
\item Propose a mechanism through which \textit{pre-migration} language skills affect \textit{current} labor market outcomes
\item Using social networks to obtain jobs is a popular method among immigrants with bad language skills
\item But using social network to obtain jobs associates with lower \textbf{Earnings} and lower \textbf{Level of job complexity}
\item No significant evidence for family migrants (\textit{even they are the main subject of pre-departure integration requirement}) \\
$\rightarrow$ Pre-migration language skill requirement is a migration restriction instrument or an integration instrument?
\end{itemize}

\end{frame}

%------------------------------------------------
\begin{frame}

\Large{\centerline{Thank you very much for your attention!}}

\end{frame}

%------------------------------------------------
\begin{frame}
\frametitle{Sensitivity analysis - Earnings}
\begin{figure}
\centering
\includegraphics[scale = 0.6]{sen_earnings.png}
  \caption{Sensitivity analysis - Outcome: Earnings}
\caption*{\small{Note: Simple mediation: AME = 0 if $\rho = -0.1$. Moderated mediation: AME = 0 if $\rho = -0.1$}}
\end{figure}
\end{frame}

%------------------------------------------------
\begin{frame}
\frametitle{Sensitivity analysis - Level of job complexity}
\begin{figure}
\centering
\includegraphics[scale = 0.6]{sen_level.png}
  \caption{Sensitivity analysis - Outcome: Job complexity}
\caption*{\small{Note: Simple mediation: AME = 0 if $\rho = -0.1$. Moderated mediation: AME = 0 if $\rho = -0.1 $}}
\end{figure}
\end{frame}
%----------------------------------------------------------------------------------------

\end{document}